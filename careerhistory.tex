\documentclass{resume}

\DeclareSymbolFont{extraup}{U}{zavm}{m}{n}
\DeclareMathSymbol{\varheart}{\mathalpha}{extraup}{86}
\DeclareMathSymbol{\vardiamond}{\mathalpha}{extraup}{87}

\usepackage{url}
\usepackage{graphicx}
\usepackage{multirow}
\usepackage[left=0.75in,top=0.6in,right=0.75in,bottom=0.6in]{geometry} % Document margins
\pagestyle{plain}

\begin{document}

  \begin{rSection}{Career history}
    \begin{rSubsection}{Qualcomm UK Ltd}{May 2014 - Present}{}{}
      %\item
      %\begin{description}
	\item[Summary]

		TODO be much more specific and detailed and include dates and literally all the projects

	  I am a member of the security team and I have contributed to writing designing and testing different kinds of software:
	  Android applications for showcasing services and libraries to our b2b customers,
	  Android services, 
	  Java and JNI libraries, 
	  C/C++ libraries such as PKCS11 compliant cryptographic libraries,
	  C/C++ system daemons,
	  trusted applications for Qualcomm proprietary TrustZone based real time operating system,
	  python programs for manipulation of binary files and for collecting analysing and visualising data

	  On the management side I was responsible for getting up to speed a new hire
	\item[Example]
	  One of my tangible contributions to the company during my stay are in the Sharp AQUOS ZETA SH-03G.
	  It is a smartphone using Qualcomm MSM8994 chipset and running Android 5.0 Lollipop. 
	  The device is equipped with a biometric authentication sensor, specifically
	  Qualcomm's 3D fingerprint technology.
	  I contributed by writing some software, i.e. a Java service, a Java library and a JNI library
	  that are integrated in the Qualcomm Snapdragon Sense ID biometrics platform and they
	  are part of an authentication and mobile payment service offered by DOCOMO (Japan's predominant mobile operator).
	  The platform can integrate as well with third-party biometric sensors and it is a certified
	  implementation of the FIDO (Fast IDentity Online) protocol.
      %\end{description}
    \end{rSubsection}

    \begin{rSubsection}{eBay International Marketing GmbH}{September 2013 - March 2014}{}{Z\"urich, Switzerland}
      %\item
	%\begin{description}
	  \item[Summary]  
	    I was an intern at eBay and I was working mostly as a web developer. 
	    We used HTML5, JavaScript and AngularJS for the front-end. 
	    The back-end used Java or Scala with the Play framework and was backed by a MySQL database or a MongoDB. 
	    Other technologies we used are: 
	    	TestNG for testing the back-end; 
		Jenkins for continuous integration; 
		Gerrit for code review; 
	    	GIT as version control system; 
		maven or sbt for build automation;
		websocket for communication among front-end and back-end.	  
	  \item[Example]
	    I am particularly proud of a project called TweetMiner, 
	    I and two other interns started this project from scratch,
	    it is a web application that uses data mining, pattern recognition and sentiment analysis techniques on tweets to find out how people feel about products they bought on eBay. 
	    %We wrote the back-end  in Scala, 
	    %the front-end in HTML5, JavaScript and AngularJS.
	    %We used TDD as development process.
	%\end{description}
    \end{rSubsection}

    \begin{rSubsection}{ISTI - National Research Council of Italy}{February 2009 - October 2009}{}{Pisa, Italy}
      \item
	I built a deductive system for an extension of RDF with fuzzy logic under the supervision of a senior researcher Umberto Straccia.
	%The system is an implementation of the deductive system described in the paper 
	%''Umberto Straccia. A Minimal Deductive System for General Fuzzy RDF. Proceedings of the 3rd International Conference on Web Reasoning and Rule Systems (RR-09), 2009.''
	The work I have done include:
	an implementation of a new language fuzzyRDF as a particular subset of RDF,
        a mechanisms of storage of RDF knowledge bases as XML files,
        a graphical and textual interface to the system,
	a proper configuration of the deductive rules of the Apache Jena reasoner,
        definition of the syntax of an apposite new query language: Fuzzy RDF Query language (FuRQL)
	and implementation in Java of a parser and a compiler from FuRQL to the SPARQL language \\ 
    \end{rSubsection}
  \end{rSection}


\end{document}
